\documentclass[11pt]{article}
\usepackage{scribe}

% Uncomment the appropriate line
%\Scribe{Your name}

\Scribes{Fill in your name}
\LectureDate{Date here}
\LectureTitle{Title here}

%\usepackage[mathcal]{euscript}


\begin{document}

\MakeScribeTop

Content here
\paragraph{This is a paragraph heading} Paragraph.

\section{This is a numbered section}
Section
\subsection*{This is an unnumbered subsection}

\bTheorem
This is a theorem.
\eTheorem

\bLemma
This is a lemma.
\Proof 
The proof comes here. 
If any lemma/theorem/corollary/example ends in an equation,
the closing box comes in the equation line like this
\[
\alpha+\beta=\gamma.\eqed
\]
\eLemmap
Do not stagger the box to be in a separate line as below.
\bLemma
This is a lemma.
\Proof 
The proof comes here. 
\[
\alpha+\beta=\gamma.
\]
\eLemma

\bCorollary
A corollary.
\eCorollary

\bExample
An example.
\eExample
'
Take a look at the \verb+scribe.sty+ file. It has several
useful definitions, such as script letters ($\cA$, $\cB$, $\ldots$);
stacked symbols like $\aeq{(a)}$, $\age{(b)}$, and so on. If you use
new definitions and macros, include it in \emph{your} source file,
and not in \verb+scribe.sty+ to prevent confusion. 

Here is an example of using all these environments.
\bTheorem Suppose $x_1\upto x_n$ and $y_1\upto y_n$ are sequences of positive numbers
both ordered in ascending or both in descending order. Then
\[
\frac{(x_1+\ldots+x_n)}n
\frac{(y_1+\ldots+y_n)}n
\le 
\frac{(x_1y_1+\ldots+x_ny_n)}n.
\]
\Proof
Start by writing
\begin{align*}
(x_1+\ldots+x_n)(y_1+\ldots+y_n)
&= \sum_{k=0}^{n-1} \sum_{i=1}^n x_i y_{((i+k)\!\!\!\!\mod n)}\\
&\le n\sum_{i=1}^n x_i y_i
\end{align*}
The last equality above follows since for all $k$ 
\[
\sum_{i=1}^n x_i y_{((i+k)\!\!\!\!\mod n)} \le \sum_{i=1}^n x_i y_{i} 
\]
because both ${x_i}$ and ${y_i}$ are both arranged in ascending (or both in descending) order.
\eTheorem


To compile, please use \verb+pdflatex+. 


\end{document}
